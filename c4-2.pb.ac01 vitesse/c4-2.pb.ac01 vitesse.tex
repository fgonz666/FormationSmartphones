\documentclass[12pt,a4paper,notitlepage]{article}
% dans ce modèle article ne pas utiliser la mention "chapter", par contre on peut
% utiliser la numérotation naturelle et les références croisées.
\usepackage[utf8x]{inputenc}
%\usepackage{ucs}
\usepackage[french]{babel}
\usepackage[T1]{fontenc}
%\usepackage{eurosym} % pour pouvoir utiliser le symbole \euro{}
\usepackage[left=17mm,right=17mm,top=17mm,bottom=17mm]{geometry}
% définition du nombre de lignes à afficher en fin ou en début de page pour
% éviter les veuves et orphelines.
\usepackage[all, defaultlines=3]{nowidow}
\usepackage{array}
\usepackage[table]{xcolor} % pour couleur de blocs, permis avec \color{couleur}{texte}

\usepackage{multirow}
\usepackage{multicol}
	\setlength{\columnsep}{0.6cm}
	\setlength{\columnseprule}{1pt}
	\def\columnseprulecolor{\color{black}}

%\usepackage{wrapfig}
% \begin{wrapfigure}[lineheight]{position = r R l L i I o O}{width}
%  minuscule = float, majuscule = force emplacement
% \end{wrapfigure}

\usepackage{hyperref} %support des url via le tag \url
\hypersetup{
	colorlinks=true,
	linkcolor=blue,
	urlcolor=purple,
	}
	\urlstyle{same}
\usepackage{amsmath}
\usepackage{amsfonts}
\usepackage{amssymb}

%\usepackage{textcomp}

\usepackage{graphicx}
%\graphicspath{ {./Images/} } % indique le chemin relatif où sont situées les images
% définition de la clé Graphic Inclusion pour paramétrer par défaut la taille des images incluses
	\setkeys{Gin}{width=0.7071\linewidth}
\usepackage{float}
	\floatplacement{table}{H} %par défaut tables là où code est posé
	\floatplacement{figure}{H} %par défaut images là où code est posé
\usepackage{siunitx}
	\sisetup{locale = FR}
%%%% Packet circuitikz permet de dessiner des circuits électriques, peut nécessiter l'utiliation de siunitx
%\usepackage[european]{circuitikz}
%\usetikzlibrary{babel}
%\usepackage{qrcode}
%\usepackage{pgfplots} %permet de tracer directement des graphiques depuis latex

\usepackage{ifthen}
%%% Gestion de la dyslexie
\newboolean{isDyslexique}
	\setboolean{isDyslexique}{false}
%%% Gestion de la correction
\newboolean{isCorrection}
	\setboolean{isCorrection}{false}

% \usepackage{palatino} %\usepackage{sans}
% \usepackage{lxfonts} %\usepackage{arev}
\ifthenelse{\boolean{isDyslexique}}{% si vrai :
	\usepackage{arev}
	\renewcommand{\baselinestretch}{1.5}
}{% si faux :
	\usepackage{lmodern}
	\renewcommand{\baselinestretch}{1.25}
}
\usepackage{lastpage}
% test de modification des headers et footers
\usepackage{fancyhdr}
\pagestyle{fancy}
\fancyhf{}
\fancyhead[LE,RO]{\rightmark}
\fancyhead[LO,RE]{\leftmark}
\fancyfoot[LE,RO]{F.S.G.}
\fancyfoot[C]{.::--- \ \thepage / \pageref*{LastPage} \ ---::.}
\fancyfoot[RE,LO]{Titre}
% fin du test de modification :)
\usepackage{tikz}
	\usetikzlibrary{babel,math}
% inclusion du paquet bclogo pour boîtes avec logo de mise en exergue
\usepackage[tikz]{bclogo}

% ========= DÉFINITION D'UN INTERLIGNE DIFFÉRENT ===============
\setlength{\parskip}{0.1cm} % définit l'espacement entre paragraphes
\renewcommand{\thesection}{\Roman{section}}

\author{F.G.}
% éviter le titre peut-être ?
\title{}

\begin{document}

\begin{flushleft}
\begin{tabular}{| m{0.15\linewidth}  m{0.8\linewidth} || }
	\hline
	\multirow{4}{*}{\includegraphics[width=\linewidth]{cycle4-logo-mvts.png}} 
	& c3-2·pc·ac
	\begin{LARGE}
		La vitesse d'un objet.
	\end{LARGE} \cr
	\cline{2-2}
	 & \cr
	 & Nom : . \ \ . \ \ . \ \ . \ \ . \ \ . \ \ . \ \ . Prénom : . \ \ . \ \ . \ \ . \ \ . \ \ . \cr
	 & \cr
	 & Classe / Groupe : . \ \ . \ \ . \ \ . Durée : ..... min. \cr
	\hline\hline
\end{tabular}
\end{flushleft}

% ======== LISTE DES IMAGES DISPONIBLES POUR CYCLES 3 ET 4 =========
% \includegraphics[scale=0.333]{cycle3-logo-elec-ondes-nrj.png}
% \includegraphics[scale=0.333]{cycle3-logo-espace.png}
% \includegraphics[scale=0.333]{cycle3-logo-mvts.png}
% \includegraphics[scale=0.333]{cycle3-logo-transfo-matiere.png}
% \includegraphics[scale=0.333]{cycle4-logo-elec-ondes-nrj.png}
% \includegraphics[scale=0.333]{cycle4-logo-espace.png}
% \includegraphics[scale=0.333]{cycle4-logo-mvts.png}
% \includegraphics[scale=0.333]{cycle4-logo-transfo-matiere.png}
\begin{flushleft}
\begin{tabular}{| m{0.03\linewidth} | m{0.75\linewidth} || m{0.015\linewidth} | m{0.015\linewidth} | m{0.015\linewidth} | m{0.015\linewidth} || }
\hline
\multirow{2}{*}{Ref} & \multirow{2}{*}{i~n~t~i~t~u~l~é~~~ d~e~~~ l~a~~~ c~o~m~p~é~t~e~n~c~e (cycle4) } & \multicolumn{4}{c ||}{É~t~a~t} \cr
	\cline{3-6}
	& & I & F & S & T \cr \hline
% A & Pratiquer des démarches scientifiques \cr \hline
%	A1 & \footnotesize{Identifier des questions de nature scientifique.} & & & & \cr \hline
%	A2 & \footnotesize{Proposer une ou des hypothèses pour répondre à une question scientifique. \newline Concevoir une expérience pour la ou les tester.} & & & & \cr \hline
	A3 & \footnotesize{Mesurer des grandeurs physiques de manière directe ou indirecte.} & & & & \cr \hline
%	A4 & \footnotesize{Interpréter des résultats expérimentaux, en tirer des conclusions et les communiquer en argumentant.} & & & & \cr \hline
%	A5 & \footnotesize{Développer des modèles simples pour expliquer des faits d'observations et mettre en \oe{}uvre des démarches propres aux sciences.} & & & & \cr \hline
% B & Concevoir, créer, réaliser \cr \hline
%	B1 & \footnotesize{Concevoir et réaliser un dispositif de mesure ou d'observation.} & & & & \cr \hline
% C & S'approprier des outils et des méthodes \cr \hline
%	C1 & \footnotesize{Effectuer des recherches bibliographiques.} & & & & \cr \hline
%	C2 & \footnotesize{Utiliser des outils numériques pour mutualiser des informations sur un sujet scientifique.} & & & & \cr \hline
	C3 & \footnotesize{Planifier une tâche expérimentale, organiser son espace de travail, garder des traces des étapes suivies et des résultats obtenus.} & & & & \cr \hline
% D & Pratiquer des langages \cr \hline
%	D1 & \footnotesize{Lire et comprendre des documents scientifiques.} & & & & \cr \hline
%	D2 & \footnotesize{Utiliser la langue française en cultivant précision, richesse de vocabulaire et syntaxe pour rendre compte des observations, expériences, hypothèses et conclusions.} & & & & \cr \hline
%	D3 & \footnotesize{S'exprimer à l'oral lors d'un débat scientifique.} & & & & \cr \hline
%	D4 & \footnotesize{Passer d'une forme de langage scientifique à une autre.} & & & & \cr \hline
% E & Mobiliser des outils numériques \cr \hline
	E1 & \footnotesize{Utiliser des outils d'acquisition et de traitement de données, de simulations et de modèles numériques.} & & & & \cr \hline
%	E2 & \scriptsize{Produire des documents scientifiques grâce à des outils numériques, en utilisant l'argumentation et le vocabulaire spécifique à la physique et à la chimie.} & & & & \cr \hline
% F & Adopter un comportement éthique et responsable \cr \hline
%	F1 & \scriptsize{Expliquer les fondements des règles de sécurité en chimie, électricité et acoustique. Réinvestir ces connaissances ainsi que celles sur les ressources et sur l'énergie, pour agir de façon responsable.} & & & & \cr	\hline
%	F2 & \footnotesize{S'impliquer dans un projet ayant une dimension citoyenne.} & & & & \cr	\hline
% G & Se situer dans l'espace et dans le temps \cr \hline
%	G1 & \footnotesize{Expliquer, par l'histoire des sciences et des techniques, comment les sciences évoluent et influencent la société.} & & & & \cr	\hline
%	G2 & \footnotesize{Identifier les différentes échelles de structuration de l'Univers.} & & & & \cr \hline
	\hline
\end{tabular}
\end{flushleft}
% ============= DÉBUT DU DOCUMENT DE TRAVAIL ==================
%\section{Rappel du cours}
%\begin{equation}
%	v = \dfrac{d}{\Delta t}
%\end{equation}
%avec :
%\begin{itemize}
%	\item d la distance parcourue en mètre ;
%	\item $\Delta t$ le temps du parcours ;
%	\item v la vitesse en mètre-par-seconde.
%\end{itemize}

\section{Expérience}
\paragraph{Étapes à suivre.} Pendant l'expérience.
\begin{enumerate}
	\item Allumez la tablette
	\item Exécutez l'application "Calculateur de Vitesse"
	\item cliquez sur l'icône en haut à droite pour calibre l'application
	\item Posez la tablette à un endroit de la table pour voir toute la règle dans l'image, mesurez la distance entre l'objet et la caméra
	\item Cliquez sur "Enregistrer"
	\item Filmez l'objet qui se déplace (il doit être visible du début à la fin sur l'écran), en parallèle l'un des membres du groupe enclenche le chronomètre au début du mouvement de l'objet et arrête ce chronomètre à la fin.
	\item Validez l'enregistrement
	\item Saisissez la distance demandée et pointez les 2 endroits (début et fin) où a été filmé l'objet.
	\item notez la vitesse (en km/h)
\end{enumerate}
\paragraph{Conseil.} Faites rouler l'objet le long de la grande règle graduée.

\section{Mesures}
Lancez l'objet à différentes vitesses et notez les différentes vitesses calculées dans le tableau qui suit.
\begin{table}
	\centering
	\rowcolors{1}{white!95!black}{white}
	\renewcommand*{\arraystretch}{1.25}
	\begin{tabular}{| l | m{0.20\linewidth} | m{0.20\linewidth} | m{0.20\linewidth} | m{0.20\linewidth} |}
		\hline
		essai & distance (m) & temps (s) & $\dfrac{distance}{temps}$ & valeur du logiciel \cr
		\hline
		1 &  &  &  &  \cr
		\hline
		2 &  &  &  &  \cr
		\hline
		3 &  &  &  &  \cr
		\hline
		4 &  &  &  &  \cr
		\hline
		5 &  &  &  &  \cr
		\hline
	\end{tabular}
\end{table}

\paragraph{Discussion.} Échange sur les valeurs obtenues et la comparaison avec le calcul issu de l'application.

\section{À retenir.}
\begin{bclogo}[couleur=pink!5, epOmbre=0.2cm, arrondi=0.1, logo=\bccoeur, nobreak=true]{La vitesse d'un objet.}
\noindent . \ \ . \ \ . \ \ . \ \ . \ \ . \ \ . \ \ . \ \ . \ \ . \ \ . \ \ . \ \ . \ \ . \ \ . \ \ . \ \ . \ \ . \ \ . \ \ . \ \ . \ \ . \ \ . \ \ . \ \ . \ \ . \ \ . \ \ . \ \ . \ \ . \ \ . \ \ . \ \ . \ \ . \ \ . \ \ . \ \ . \ \ . \ \ . \ \ . \ \ . \ \ . \ \ . \ \ . \ \ . \ \ . \ \ . \ \ . \ \ . \ \ . \ \ . \ \ . \ \ . \ \ . \ \ . \ \ . \ \ . \ \ . \ \ . \ \ . \ \ . \ \ . \ \ . \ \ . \ \ . \ \ . \ \ . \ \ . \ \ . \ \ . \ \ . \ \ . \ \ . \ \ . \ \ . \ \ . \ \ . \ \ . \ \ . \ \ . \ \ . \ \ . \ \ . \ \ . \ \ . \ \ . \ \ . \ \ . \ \ . \ \ . \ \ . \ \ . \ \ . \ \ . \ \ . \ \ . \ \ . \ \ . \ \ . \ \ . \ \ . \ \ . \ \ . \ \ . \ \ . \ \ . \ \ . \ \ . \ \ . \ \ . \ \ . \ \ . \ \ . \ \ . \ \ . \ \ . \ \ . \ \ . \ \ . \ \ . \ \ . \ \ . \ \ . \ \ . \ \ . \ \ . \ \ . \ \ . \ \ . \ \ . \ \ . \ \ . \ \ . \ \ . \ \ . \ \ . \ \ . \ \ . \ \ . \ \ . \ \ . \ \ . \ \ . \ \ . \ \ . \ \ . \ \ . \ \ . \ \ . \ \ . \ \ . \ \ . \ \ . \ \ . \ \ . \ \ . \ \ . \ \ . \ \ . \ \ . \ \ . \ \ . \ \ . \ \ . \ \ . \ \ . \ \ . \ \ . \ \ . \ \ . \ \ . \ \ . \ \ . \ \ . \ \ .
\end{bclogo}

\end{document}

%%%%%%%%%%%%%%%%%%%%%%%%%%%%%%%%%%%%%%%%%%%%%%%%%%%%%%%

% cadre pour le cours / à savoir par coeur.
\begin{bclogo}[couleur=pink!5, epOmbre=0.2cm, arrondi=0.1, logo=\bccoeur, nobreak=true]{}

\end{bclogo}

% cadre pour un document à analyser (informatif)
\begin{bclogo}[couleur=blue!5, epOmbre=0.2cm, arrondi=0.1, logo=\bcinfo, nobreak=true]{}

\end{bclogo}

% cadre pour le travail :
\begin{bclogo}[couleur=yellow!5, arrondi=0.1, logo=\bccrayon, nobreak=true]{}

\end{bclogo}
